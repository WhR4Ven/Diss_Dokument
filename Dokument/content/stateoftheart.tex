\subsection*{Stand der Technik}

In diesem Kapitel werden die aktuelle technische Lösungen im Bereich Cloud Computing und verteiltes Rechnen
 vorgestellt, sowie die gängigsten Softwarewerkzeuge mit denen diese Systeme erzeugt und verwaltet werden können.

\subsection*{Cloud Computing}
Cloud Computing hat in der Literatur mehrere Definitionen. \cite{Marston2011} Die am meisten verbreitete Definition
wurde vom National Institute of Standards and Technology (NIST) festgelegt.
Nach dieser Definition ist Cloud Computing ein Modell, das einen allgegenwärtigen,
bequemen, bedarfsgerechten Zugang zu einem gemeinsamen Pool an konfigurierbarer Rechenressourcen
die jederzeit und von jedem Ort aus über das Internet oder ein Netzwerk schnell zur Verfügung gestellt 
werden können \cite{Mell2011} 

Cloud Computing ermöglicht Organisationen rechenkapazitäten zu erhalten, die sie selber nicht anschaffen und 
betreiben können.\cite{Arasaratnam2011}

Es hat sich als dominantes Geschäftsmodell in der IT Infrastruktur durchgesetzt. \cite{Benlian2018}

Characteristiken:
    "On-Demand": Kunden werden Rechenkapazitäten automatisch bereitbestellt, ohne menschliche Interaktion
    "Breiter Netzzugang:" 
    "Ressourcen-Pooling:" 
    "Schnelle Elastizität:" 
    "Measured Service:" 
\clearpage