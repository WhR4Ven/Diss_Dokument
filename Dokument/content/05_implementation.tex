\chapter{Implementierung}

Dieses Kapitel beschreibt die Implementierung von Runtime Node und Master und vergleicht die Scheduling-Methoden.

\section{Implementierung Runtime Node}


\subsection{Implementierung der Applikationsisolierung}
Ausgehend von den Anforderungen an die Komponente Runtime Node ist eine möglichst hohe Kompatibilität mit verschiedenen Betriebssystemen zu gewährleisten. In der \autoref{Methode Softwareplattform und Scheduling} wurde bereits festgelegt, dass Runtime Node die Isolationsmethoden des Linux-Kernels verwendet, um die Anwendungen zu isolieren. Für Unix-basierte Betriebssysteme wie Linux wurde der \gls{POSIX}-Standard entwickelt, um die Kompatibilität von Software zu erhöhen, die auf Betriebssystemfunktionen zugreift. Er definiert eine einheitliche \gls{API} für Kommandozeilen- und Shell-Funktionalität. Um die Kompatibilitätsanforderungen zu erfüllen, verwendet die Runtime Node Komponente die \gls{POSIX} \gls{API} Funktionen zur Isolierung von Anwendungen. Unix-basierte Betriebssystemkernel wie der von Linux sind in der Programmiersprache C implementiert, weshalb die \gls{POSIX} \gls{API} auch aus standardisierten C-Funktionen besteht. Für eine einfache Integration der \gls{API} und eine hohe Anzahl an verfügbaren Compiler für verschiedene Architekturen wurde die 