\chapter{Einleitung}

Autonome Fahrzeuge zählen zu den bedeutendsten technologischen Entwicklungen im Bereich der Mobilität unserer Zeit. Angetrieben durch Fortschritte in den Bereichen Sensorik, künstliche Intelligenz und Fahrzeugarchitektur ist absehbar, dass autonome Fahrfunktionen zunehmend in Serienfahrzeuge integriert werden. Dabei spielen vielfältige gesellschaftliche und ökonomische Faktoren eine Rolle: Neben dem demografischen Wandel und dem Fachkräftemangel im Transportsektor ist vor allem die fortschreitende Urbanisierung von zentraler Bedeutung. Mit der zunehmenden Konzentration der Bevölkerung in städtischen Räumen steigen sowohl das Verkehrsaufkommen als auch die Anforderungen an Sicherheit, Effizienz und Nachhaltigkeit im Mobilitätssystem. Laut einer Erhebung des United Nations Department of Economic and Social Affairs \cite{bpb2017} werden bis zum Jahr 2050 mehr als 50 \% der Menschen in den ökonomisch entwickelten Ländern in Städten leben.

Autonome Fahrzeuge werden in diesem Zusammenhang als Schlüsseltechnologie betrachtet, da sie einen Beitrag zur Reduzierung von Unfällen, zur Optimierung des Verkehrsflusses und zur besseren Auslastung vorhandener Infrastrukturen leisten können.