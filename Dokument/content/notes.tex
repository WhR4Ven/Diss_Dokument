\begin{notes}
    \item Netzwerkaufbau
    \begin{notes}
        \item Edge Computing (zentrale Verwaltung)
        \begin{notes}
            \item “a form of distributed computing in which processing and storage takes place on a set of networked machines which are near the edge, where the nearness is defined by the system’s requirements”  (ISO/IEC: Tr 30164:2020 - internet of things (iot) -edge computing. Tech. rep., ISO/IEC (2020))
            \item Motivation: Latenzreduzierung, unbenutzte Resourcen verwenden
            \item Fahrzeuge sind Edge Server
            \item kann mit fest installierten Edge Server kombiniert werden
            \item Basissatation zuständig für Verbindung zwischen lokale Edge server und Cloud Netzwerk 
        \end{notes}
        \item Fog Computing (lokale Fahrzeuge schließen sich zu Rehchenknoten zusammen)
        \item “a horizontal system-level architecture that distributes computing, storage, and networking functions closer to the user along a cloud-to-thing continuum”  OpenFog Consortium (Group, O.C.A.W., et al.: Openfog reference architecture for fog computing. OPFRA001 20817, 162 (2017))
        \item Mist Computing
        \item Herausforderungen:
        \begin{notes}
            \item allgemeine Rechenaufgaben auf spezialisierte Hardware
            \item Erkennung von Edge nodes
            \item Task auslagerung und Verteilung
            \item Keine Beeinträchtigumg der Funktionalität des Edge Gerätes (z.b. Überlastung)
            \item Sicherheit
        \end{notes}
    \end{notes}
    \item Kommunikation
    \begin{notes}
        \item Zertifikate (Public/Private Key)
        \item identitätsverschlüsselung
        \item Belohnung für bereitgestellte Rechenleistung
    \end{notes}
    \item Resourcenverteilung
    \begin{notes}
        \item Bestimmung der verfügbaren Rechenleistung 
        \item Optimierungsalgorythmen
        \item Stackelberg Model
    \end{notes}
    \item Publish Subscribe
    \begin{notes}
        \item 
        \item Optimierungsalgorythmen
        \item Stackelberg Model
    \end{notes}
    \item Softwarearchitektur in Fahrzeugen
   \begin{notes}
        \item RTOS
        \item Moddle Layer (ROS, keine automotive alternative stand 2019)
        \item Cloud
    \end{notes}

\end{notes}