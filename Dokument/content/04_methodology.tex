\chapter{Gesamtkonzept Softwareplattform}

In diesem Kapitel wird anhand der Anforderungen aus dem Forschungsansatz eine Gesamtarchitektur des Softwareplattforms vorgestellt, sowie die Softwarearchitektur der beiden Softwarekomponenten \enquote{Runtime Node} und \enquote{Runtime Master}. 

\section{Verteiltes Rechenplattform aus autonomen Fahrzeugsteuergeräten}

Die Rechenleistung von autonomen Fahrzeugen kann in verschiedenen Nutzungskonzepten verwendet werden. Im Kapitel \autoref{Hardwareumgebung in Fahrzeugen} wurde die Hardware vorgestellt, die typischerweise in autonomen Fahrzeugsteuergeräten zu finden ist. Es wird deutlich, dass diese Steuergeräte über Hardwaremodule verfügen, die die Rechenleistung für spezifische Rechenoperationen erhöhen. Grafikprozessoren beinhalten viele Rechenkerne, die parallel auf gro0en Datenmengen opreationen ausführen können. In den meisten Architekturen können die GPU-Cores nur in Gruppen verwendet werden, wobei jede Gruppe die gleiche Rechenoperation ausführen muss. Kerne für Neuronale Netzwerk Berechnungen implementieren Matrix Operationen in Hardware, so dass diese in erheblich weniger Rechenzyklen als bei sequentieller Berechnung durchgeführt werden können. 