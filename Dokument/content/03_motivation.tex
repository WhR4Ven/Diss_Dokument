\chapter{Forschungsbedarf und -ansatz}

Ziel dieser Arbeit ist die Erarbeitung eines Konzepts für die Nutzung von Rechenressourcen in autonomen Fahrzeugen. Ein solches Konzept soll möglichst allgemein verwendbar sein, mit möglichst geringer Abhängigkeit zur eingesetzter Hardware und Betriebssystem. Zudem soll die Kommunikationsschnittstelle flexibel gestaltet werden, um die Integration in bestehenden Kommunikationsprotokolle zu vereinfachen. Hierfür wird eine entsprechend geeignete Struktur vorgesehen mit definierten Schnittstellen. In dieser Arbeit werden die Anforderungen ermittelt, die für die Nutzung von Rechenressourcen in Fahrzeugen relevant sind. Anhand der ermittelten Anforderungen werden die vorhandenen technische Lösungen bewertet und ein geeigneter Ansatz und Architektur erarbeitet.



\section{Forschungsbedarf}

Im folgenden Abschnitt wird der Forschungsbedarf anhand der Stand der Technik aus Kapitel 2 vorgestellt. 

\subsection{Virtualisierung}

\subsection{Orchestrierung}

