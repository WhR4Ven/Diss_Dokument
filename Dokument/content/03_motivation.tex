\chapter{Forschungsbedarf und -ansatz}

Ziel dieser Arbeit ist die Erarbeitung eines Konzepts für die Nutzung von Rechenressourcen in autonomen Fahrzeugen. Ein solches Konzept soll möglichst allgemein verwendbar sein, mit möglichst geringer Abhängigkeit zur eingesetzter Hardware und Betriebssystem. Zudem soll die Kommunikationsschnittstelle flexibel gestaltet werden, um die Integration in bestehenden Kommunikationsprotokolle zu vereinfachen. Hierfür wird eine entsprechend geeignete Struktur vorgesehen mit definierten Schnittstellen. In dieser Arbeit werden die Anforderungen ermittelt, die für die Nutzung von Rechenressourcen in Fahrzeugen relevant sind. Anhand der ermittelten Anforderungen werden die vorhandenen technische Lösungen bewertet und ein geeigneter Ansatz und Architektur erarbeitet.

\section{Forschungsbedarf}

Im folgenden Abschnitt wird der Forschungsbedarf anhand der Stand der Technik aus Kapitel 2 vorgestellt. Hierfür werden die Methoden für Applikationsmanagement betrachtet, die in IT Infrastrukturen verwendet werden und deren Eignung für den Einsatz in Fahrzeugen. 

\subsection{Virtualisierung}
Im \autoref{Applikationsmanagement} wurden die gängigen Virtualisierungsmethoden vorgestellt, die vorwiegend in IT Infrastruktur zur Verwendung kommen. Diese bilden die Grundlage für Applikationsmanagement in IT Systemen. Diese Methoden können grundsätzlich auch für Fahrzeugsteuergeräte verwendet werden, allerdings unterscheiden sich die Steuergeräte in Hardwarekomponenten und Betriebssystem, so dass die vorhandene Lösungen aus dem IT Bereich nicht unmittelbar verwendet werden können. Um eine möglichst hohe Kompatibilität für verschiedene Architekturen zu gewährleisten, müssen die Virtualisierungsmethoden für verschiedene Architekturen verfügbar oder portierbar sein. Vorteilhaft sind hierbei Lösungen, die mit offenem Quellcode, da hier die Portierung je nach Lizenzbedingungen auch unabhängig vom Entwickler vorgenommen werden kann. Im Allgemeinen muss eine Virtualisierungstechnologie, die auf verschiedenen Plattformen zum Einsatz kommen soll, eine einfache Portierbarkeit gewährleisten. Die aktuellen Lösungen aus der IT gewährleisten die Kompatibilität mit gängiger Hardwareplattformen sowie Betriebssystemen. Die Portierbarkeit auf Fahrzeugsteuergeräte stand bei der Entwicklung nicht im Vordergrund. Virtuelle Maschinen erfordern die hardwareseitige Unterstützung von Virtualisierungsmethoden, so dass diese art von Virtualisierung schwieriger zu portieren ist. Containerisierung nutzt bestimmte Linux Kernelfeatures, so dass hier eine Einschränkung seitens Betriebssystem auf dem Zielgerät stattfindet.
\subsection{Orchestrierung}
Für die Orchestrierung von virtuellen Umgebungen existiert eine Vielzahl von Anwendungen die im \autoref{Orchestrierung Stand der Technik} vorgestellt wurden. Aus \autoref{Verteiltes Rechnen in Fahrzeugen} geht hervor, dass die Orchestrierung von Applikationen in Fahrzeugen ein aktuelles Forschungsfeld ist, wo stand jetzt (2024) keine fertige kommerzielle Lösungen existieren. Aus funktionaler Sicht sind die Orchestrierungsapplikationen aus der IT auch für die Orchestrierung von Fahrzeugapplikationen geeignet. Ein wesentlicher Unterschied zwischen der Orchestrierung verteilter IT-Systeme und Fahrzeugen besteht in der Dynamik, mit der sich die Eigenschaften der Systeme ändern. Verteilte Systeme im IT Bereich bestehen üblicherweise aus einzelnen Server oder Rechencluster, die verwaltet werden. Die Verfügbarkeit der Systeme ist hoch, sie gehen nur im Fehlerfall oder bei geplanten Wartungen offline. Die verfügbaren Rechenressourcen sind ebenfalls einfach kalkulierbar, da die Systeme in der Regel nur die Applikationen ausführen, die sie vom Orchestrierer erhalten. Die Orchestrierung der ungenutzten Rechenressourcen von Fahrzeugen ist im Vergleich mit einem signifikanten Mehraufwand verbunden. 
 
\subsection{Architektur verteilter Systeme}

\subsection{Anforderungsanalyse}


