\chapter{Forschungsbedarf und -ansatz}

Ziel dieser Arbeit ist die Erarbeitung eines Konzepts für die Nutzung von Rechenressourcen in autonomen Fahrzeugen. Ein solches Konzept soll möglichst allgemein verwendbar sein, mit möglichst geringer Abhängigkeit zur eingesetzter Hardware und Betriebssystem. Zudem soll die Kommunikationsschnittstelle flexibel gestaltet werden, um die Integration in bestehenden Kommunikationsprotokolle zu vereinfachen. Hierfür wird eine entsprechend geeignete Struktur vorgesehen mit definierten Schnittstellen. In dieser Arbeit werden die Anforderungen ermittelt, die für die Nutzung von Rechenressourcen in Fahrzeugen relevant sind. Anhand der ermittelten Anforderungen werden die vorhandenen technische Lösungen bewertet und ein geeigneter Ansatz und Architektur erarbeitet.



\section{Forschungsbedarf}

Im folgenden Abschnitt wird der Forschungsbedarf anhand der Stand der Technik aus Kapitel 2 vorgestellt. 

\subsection{Virtualisierung}
Im \autoref{Applikationsmanagement} wurden die gängigen Virtualisierungsmethoden vorgestellt, die vorwiegend in IT Infrastruktur zur Verwendung kommen. Diese Methoden können grundsätzlich auch für Fahrzeugsteuergeräte verwendet werden, allerdings unterscheiden sich die Steuergeräte in Hardwarekomponenten und Betriebssystem, so dass die vorhandene Lösungen aus dem IT Bereich nicht unmittelbar verwendet werden können. Um eine möglichst hohe Kompatibilität für verschiedene Architekturen zu gewährleisten, müssen die Virtualisierungsmethoden für verschiedene Architekturen verfügbar oder portierbar sein. Vorteilhaft sind hierbei Lösungen, die mit offenem Quellcode, da hier die Portierung je nach Lizenzbedingungen auch unabhängig vom Entwickler vorgenommen werden kann. Im Allgemeinen muss eine Virtualisierungstechnologie, die auf verschiedenen Plattformen zum Einsatz kommen soll, eine einfache Portierbarkeit gewährleisten. Die aktuellen Lösungen aus der IT gewährleisten die Kompatibilität mit gängiger Hardwareplattformen sowie Betriebssystemen. Die Portierbarkeit auf Fahrzeugsteuergeräte stand bei der Entwicklung nicht im Vordergrund.
\subsection{Orchestrierung}
\autoref{Orchestrierung Stand der Technik}

\subsection{Architektur verteilter Systeme}

\subsection{Anforderungsanalyse}


